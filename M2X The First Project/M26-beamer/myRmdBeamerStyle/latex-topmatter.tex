% Hangul in rchunk 2019-02-12  
% Not the output yet
\usepackage[hangul]{kotex}
\usepackage{pdflscape}
\usepackage{colortbl}
% \usepackage[table]{xcolor}
\newcolumntype{U}{>{\columncolor[gray]{0.8}}c}
\usepackage{tabularx,booktabs}
\usepackage{boxedminipage}
\usepackage{graphicx}
\usepackage{rotating}
\usepackage{rotfloat}
\usepackage{booktabs}
\usepackage{longtable}
\usepackage{subfigure}
\usepackage{wrapfig}
\usepackage{multirow}

% declare overall beamer theme to use as baseline
\usetheme{default}
\setbeamertemplate{navigation symbols}{} 
\setbeamertemplate{footline}[frame number]

% make code-output smaller (2018-05-23)
% CODE CHUNK FONT
\DefineVerbatimEnvironment{Highlighting}{Verbatim}{fontsize=\footnotesize,commandchars=\\\{\}}

% two columns layout (2018-05-23)
% \def\begcols{\begin{columns}}
% \def\endcols{\end{columns}}
% \def\begcol{\begin{column}}
% \def\endcol{\end{column}}
% left column; right column; end column.
\def\lc{\begin{columns} \begin{column}{.48\textwidth}}
\def\rc{\end{column} \begin{column}{.48\textwidth}}
\def\ec{\end{column} \end{columns}}

% vspace shortcut (2018-05-23)
\def\br{\vspace{1pt}} % line break

% make console-output footnotesize (2018-05-23)
\makeatletter
\def\verbatim{\footnotesize\@verbatim \frenchspacing\@vobeyspaces \@xverbatim}
\makeatother


\setlength{\parskip}{0pt}


\setlength{\OuterFrameSep}{-4pt}
\makeatletter
\preto{\@verbatim}{\topsep=-5pt \partopsep=-5pt }
\makeatother